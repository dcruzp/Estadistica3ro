\documentclass{article}

\usepackage[utf8]{inputenc}
\usepackage[spanish]{babel}
\usepackage{graphicx}
\usepackage{hyperref}
\usepackage{cancel}
\usepackage{amsmath} % Este paquete define bmatrix
\usepackage{multirow, array} % para las tablas
\usepackage{float} % para usar [H]
\usepackage{array}
\usepackage{minted}
\usepackage{color}

\definecolor{bg}{rgb}{0.95,0.95,0.95}

\begin{document}
	\title{Estad\'istica.  Seminario 2}
	\author{Daniel de la Cruz Prieto} 
	\maketitle
	
	\section*{Ejercicio 1 Clase Pr\'actica 6 }
	
	\begin{table}[ht]
		\centering
		\caption{Tabla de Datos}
		\begin{tabular}{rrrrr}
			\hline
			x & y & xy & xx & yy \\ 
			\hline
			-3.00 & 14.00 & -42.00 & 9.00 & 196.00 \\ 
			-1.00 & 4.00 & -4.00 & 1.00 & 16.00 \\ 
			1.00 & 2.00 & 2.00 & 1.00 & 4.00 \\ 
			3.00 & 8.00 & 24.00 & 9.00 & 64.00 \\ 
			5.00 & 22.00 & 110.00 & 25.00 & 484.00 \\ 
			\hline
			7.00 & 44.00 & 308.00 & 49.00 & 1936.00 \\ 
			\hline
		\end{tabular}
	\end{table}
	
	\section*{Ejercicio 6 Clase Pr\'actica 6 }
\end{document}